% Sizes
\pgfmathsetmacro{\cardwidth}{6.3}
\pgfmathsetmacro{\cardheight}{8.8}

\pgfmathsetmacro{\verticalspacing}{0.4}
\pgfmathsetmacro{\horizontalspacing}{0.4}

\pgfmathsetmacro{\abilitywidth}{\cardwidth-2*\horizontalspacing}
\pgfmathsetmacro{\fullabilityheight}{\cardheight-2*\verticalspacing}
\pgfmathsetmacro{\halfabilityheight}{0.5*(\cardheight-3*\verticalspacing)}
\pgfmathsetmacro{\abilitybannerwidth}{1}
\pgfmathsetmacro{\abilityiconsize}{0.5}
\pgfmathsetmacro{\abilitytextwidth}{\abilitywidth-\abilitybannerwidth-0.4}

% Shapes
\def\shapeCard{(0,0) rectangle (\cardwidth,\cardheight)}

\tikzset{
	cardcorners/.style={
		rounded corners=0.2cm
	}
}

% Commands
\newcommand{\cardborder}{
	\draw[lightgray,cardcorners] \shapeCard;
}

\newcommand{\xcardcredit}[1]{
	#1
}

\newcommand{\topability}[3]{
	\fill[#3, cardcorners] (\horizontalspacing, \cardheight-\verticalspacing) rectangle (\horizontalspacing+\abilitybannerwidth, \cardheight-\verticalspacing-\halfabilityheight);
	\fill[#3] (\horizontalspacing+0.5*\abilitybannerwidth, \cardheight-\verticalspacing) rectangle (\horizontalspacing+\abilitybannerwidth, \cardheight-\verticalspacing-\halfabilityheight);
	\draw[black, cardcorners] (\horizontalspacing, \cardheight-\verticalspacing) rectangle (\cardwidth-\horizontalspacing, \cardheight-\verticalspacing-\halfabilityheight);
	\node (abilityicon) at (\horizontalspacing+\abilitybannerwidth+0.5*\abilityiconsize,\cardheight-\verticalspacing-0.5*\abilityiconsize) {\includegraphics[width=\abilityiconsize cm]{#2}};
	\node [text width=\halfabilityheight cm, rotate=90] at (\horizontalspacing+0.5*\abilitybannerwidth, \cardheight-\verticalspacing-0.5*\halfabilityheight) {
		\begin{center}
			\begin{center}\color{white}{\Large \bf #1}\end{center}
		\end{center}
	};
}
\newcommand{\topabilitytext}[1]{
	\node [text width=\abilitytextwidth cm] at (0.5*\cardwidth+0.5*\abilitybannerwidth, \cardheight-\verticalspacing-0.5*\halfabilityheight) {
		\begin{center}
			\small #1
		\end{center}
	};
}

\newcommand{\bottomability}[3]{
	\fill[#3, cardcorners] (\horizontalspacing, \verticalspacing) rectangle (\horizontalspacing+\abilitybannerwidth, \verticalspacing+\halfabilityheight);
	\fill[#3] (\horizontalspacing+0.5*\abilitybannerwidth, \verticalspacing) rectangle (\horizontalspacing+\abilitybannerwidth, \verticalspacing+\halfabilityheight);
	\draw[black, cardcorners] (\horizontalspacing, \verticalspacing) rectangle (\cardwidth-\horizontalspacing, \verticalspacing+\halfabilityheight);
	\node (abilityicon) at (\horizontalspacing+\abilitybannerwidth+0.5*\abilityiconsize,\verticalspacing+\halfabilityheight-0.5*\abilityiconsize) {\includegraphics[width=\abilityiconsize cm]{#2}};
	\node [text width=\halfabilityheight cm, rotate=90] at (\horizontalspacing+0.5*\abilitybannerwidth, \verticalspacing+0.5*\halfabilityheight) {
		\begin{center}
			\begin{center}\color{white}{\Large \bf #1}\end{center}
		\end{center}
	};
}
\newcommand{\bottomabilitytext}[1]{
	\node [text width=\abilitytextwidth cm] at (0.5*\cardwidth+0.5*\abilitybannerwidth, \verticalspacing+0.5*\halfabilityheight) {
		\begin{center}
			\small #1
		\end{center}
	};
}

\newcommand{\fullability}[3]{
	\fill[#3, cardcorners] (\horizontalspacing, \cardheight-\verticalspacing) rectangle (\horizontalspacing+\abilitybannerwidth, \cardheight-\verticalspacing-\fullabilityheight);
	\fill[#3] (\horizontalspacing+0.5*\abilitybannerwidth, \cardheight-\verticalspacing) rectangle (\horizontalspacing+\abilitybannerwidth, \cardheight-\verticalspacing-\fullabilityheight);
	\draw[black, cardcorners] (\horizontalspacing, \cardheight-\verticalspacing) rectangle (\cardwidth-\horizontalspacing, \cardheight-\verticalspacing-\fullabilityheight);
	\node (abilityicon) at (\horizontalspacing+\abilitybannerwidth+0.5*\abilityiconsize,\cardheight-\verticalspacing-0.5*\abilityiconsize) {\includegraphics[width=\abilityiconsize cm]{#2}};
	\node [text width=\fullabilityheight cm, rotate=90] at (\horizontalspacing+0.5*\abilitybannerwidth, \cardheight-\verticalspacing-0.5*\fullabilityheight) {
		\begin{center}
			\begin{center}\color{white}{\Large \bf #1}\end{center}
		\end{center}
	};
}
\newcommand{\fullabilitytext}[1]{
	\node [text width=\abilitytextwidth cm] at (0.5*\cardwidth+0.5*\abilitybannerwidth, \verticalspacing+0.5*\fullabilityheight) {
		\begin{center}
			\small #1
		\end{center}
	};
}
